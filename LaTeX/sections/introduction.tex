\section{Introduction}
\label{sec:introduction}

Scintillators are frequently used for measuring the cross section of reactions relevant in astrophysics. This type of detectors provide high efficiencies at the cost of not optimal energy resolution and internal background, compared to usual $\gamma$-spectroscopy detectors as HPGe.  \\

The working principle of scintillators is the following. When stricken by gamma-ray, the primary electrons raise secondary electrons to the conduction band, leaving holes in the valence band. 
If the electrons are allowed to de-excite by falling back to the valence band, they will emit electromagnetic radiation, and if this radiation is in, or near, optical wavelengths, it can be detected by a photomultiplier or other lightmeasuring
device to provide the detector signal.

It may happen that the band gap is large and photons emitted by de-excitation of electrons directly from the conduction band would be far outside of the
visible range, making the detection of the light difficult. Furthermore, the bulk of the material absorbs the emitted photons before they reach the photomultiplier. Both problems are solved by using an activator. The ground state of these activator sites lies just above the valence band and the excited states somewhat below the conduction band. This means that the photon energy released when these levels de-excite will be lower and the electromagnetic radiation will be of a longer wavelength, perhaps in the visible range. It also means that the emission wavelength will no longer match the absorption characteristics of the scintillator and so much less light will be lost before measurement by the
photomultiplier. \\


The ideal scintillator material for gamma-ray
detection and spectrometry must have the following properties:
\begin{itemize}
    
    \item there must be a reasonable number of electron–hole pairs produced per unit of gamma-ray energy;
    
    \item it would be very desirable for the material to have a high stopping power for gamma radiation (which means, in practice, high density and atomic number);

    \item for spectrometry, the response must be proportional to energy;

    \item the scintillator must be transparent to the emitted light;

    \item the decay time of the excited state must be short to allow high count rates;

    \item the material should be available in optical quality in reasonable amounts at reasonable cost;

    \item the refractive index of the material should be near to that of glass (ca. 1.5) to permit efficient coupling to photomultipliers.
    
\end{itemize}


In this work, two $LaBr_3(Ce)$ (Cerium activated Bromide of Lanthanum) detectors will be studied. This material provides several advantages over the more common $NaI(Tl)$, namely a greater light output, better resolution, shorter scintillation decay time and higher efficiency due to its higher density.
A single disadvantage is its inherent radioactive impurity content due to Lanthanum's naturally
occurring radioisotope, $^{138}La$. \cite{gilmore_2008}\\

The two detectors will be first calibrated using three different radioactive sources.
Afterward, we will study their energy resolution and timing performances in coincidence. Lastly, their efficiency will be computed using different methods.